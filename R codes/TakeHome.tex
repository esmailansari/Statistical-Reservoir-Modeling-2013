%%%%%%%%%%%%%%%%%%%%%%%%%%%%%%%%%%%%%%%%%
% Short Sectioned Assignment
% LaTeX Template
% Version 1.0 (5/5/12)
%
% This template has been downloaded from:
% http://www.LaTeXTemplates.com
%
% Original author:
% Frits Wenneker (http://www.howtotex.com)
%
% License:
% CC BY-NC-SA 3.0 (http://creativecommons.org/licenses/by-nc-sa/3.0/)
%
%%%%%%%%%%%%%%%%%%%%%%%%%%%%%%%%%%%%%%%%%

%----------------------------------------------------------------------------------------
%  PACKAGES AND OTHER DOCUMENT CONFIGURATIONS
%----------------------------------------------------------------------------------------

\documentclass[paper=a4, fontsize=12pt]{scrartcl} % A4 paper and 11pt font size

\usepackage[T1]{fontenc} % Use 8-bit encoding that has 256 glyphs
\usepackage{fourier} % Use the Adobe Utopia font for the document - comment this line to return to the LaTeX default
\usepackage[english]{babel} % English language/hyphenation
\usepackage{amsmath,amsfonts,amsthm} % Math packages

\usepackage{lipsum} % Used for inserting dummy 'Lorem ipsum' text into the template
\usepackage{pdfpages}
\usepackage{sectsty} % Allows customizing section commands
\allsectionsfont{ \normalfont\scshape} % Make all sections centered, the default font and small caps
\usepackage{algorithmic}
\usepackage{algorithm}
\usepackage{fancyhdr} % Custom headers and footers
\pagestyle{fancyplain} % Makes all pages in the document conform to the custom headers and footers
\fancyhead{} % No page header - if you want one, create it in the same way as the footers below
\fancyfoot[L]{} % Empty left footer
\fancyfoot[C]{} % Empty center footer
\fancyfoot[R]{\thepage} % Page numbering for right footer
\renewcommand{\headrulewidth}{0pt} % Remove header underlines
\renewcommand{\footrulewidth}{0pt} % Remove footer underlines
\setlength{\headheight}{13.6pt} % Customize the height of the header

\numberwithin{equation}{section} % Number equations within sections (i.e. 1.1, 1.2, 2.1, 2.2 instead of 1, 2, 3, 4)
\numberwithin{figure}{section} % Number figures within sections (i.e. 1.1, 1.2, 2.1, 2.2 instead of 1, 2, 3, 4)
\numberwithin{table}{section} % Number tables within sections (i.e. 1.1, 1.2, 2.1, 2.2 instead of 1, 2, 3, 4)

\setlength\parindent{1pt} % Removes all indentation from paragraphs - comment this line for an assignment with lots of text

%----------------------------------------------------------------------------------------
%	TITLE SECTION
%----------------------------------------------------------------------------------------

\newcommand{\horrule}[1]{\rule{\linewidth}{#1}} % Create horizontal rule command with 1 argument of height

\title{	
\normalfont \normalsize 
\textsc{Statistical Reservoir Modeling \\
(PETE 7285)} \\ [25pt] % Your university, school and/or department name(s)
\horrule{0.5pt} \\[0.4cm] % Thin top horizontal rule
\huge  Exam 1 \\ % The assignment title
\horrule{2pt} \\[0.5cm] % Thick bottom horizontal rule
}

\author{Esmail Ansari} % Your name

\date{\small\today} % Today's date or a custom date

\usepackage{Sweave}
\begin{document}
\input{TakeHome-concordance}

\maketitle % Print the title

%----------------------------------------------------------------------------------------
%	PROBLEM 1
%----------------------------------------------------------------------------------------

\section{Integrating models for facies prediction}

%-----------------------------------------
%----------------------------------------------------
%-------------
\subsection{Part A}
The course codes were trimmed so that they only output scaled kapur data (function \texttt{scaleKapur}). Then, the kapur facies are counted using \texttt{table} function and probability (proportion) of each facies is computed. 

\begin{Schunk}
\begin{Sinput}
> #cleaning variables and closing plots
> rm(list=ls())
> if (!is.null(dev.list())) dev.off()
> #loading packages
> require(boot)
> require(nnet)
> require(RColorBrewer)
> require(ggplot2)
> require(lattice)
> line.colors <- brewer.pal(N.facies,"Dark2")
> #Counting facies using 'table' function
> facies     = kapur["facies"]
> tab.facies = table(facies)
> N.facies   = length(tab.facies)
> tot.obs    = sum(tab.facies)
> p.facies   = tab.facies/tot.obs
\end{Sinput}
\end{Schunk}

For fitting a beta distribution to the facies, the total counts are used for estimating alpha and beta. \texttt{dbeta} function is then used for calculating the probability of facies proportions. The results are then tabulated to be used by \textit{lattice} package. 

\begin{Schunk}
\begin{Sinput}
> for (i in 1:N.facies){
+   #evaluating the parameters of beta function and facies pdf
+   x     =  seq(0,1,0.001)
+   alpha = tab.facies+1
+   beta  =  tot.obs-tab.facies+1
+   p.df  = dbeta(x,alpha[i],beta[i])
+   
+   #tidy up for lattice package 
+   current.size = length(x)
+   sum.size.pri = current.size+sum.size.pri
+   all.prior[(sum.size.pri-current.size):(sum.size.pri-1),1] = x
+   all.prior[(sum.size.pri-current.size):(sum.size.pri-1),2] = 
+     p.df/sum(p.df)
+   all.prior[(sum.size.pri-current.size):(sum.size.pri-1),3] =
+    rep(paste("facies ",i),length(x))    
+ }
> #plotting priors using lattice package 
> line.colors = brewer.pal(N.facies,"Dark2")
> myStrip     = function(which.panel, factor.levels, ...) {
+   panel.rect(0, 0, 1, 1,col = line.colors[which.panel],border = 1)
+   panel.text(x = 0.5, y = 0.5,font = 2,lab = factor.levels[which.panel])
+ }
> print(xyplot(Probability ~ Proportion|Facies.type,
+                data = all.prior[1:sum.size.pri-1,],
+                type = "l",lwd = 3,strip = myStrip))
\end{Sinput}
\end{Schunk}


%-------------
\subsection{Part B}
Function \texttt{prob.facies} was prepared for bootstraping using \texttt{boot} function. This function fits the multinomial regression to the facies and outputs the probability of probability of each facies. 
\begin{Schunk}
\begin{Sinput}
> #Defining the probability function to be bootstrapped
> prob.facies     <-  function(formula, data, indices){
+   selected.data   =  data[indices,]
+   kapur.glm       =  multinom(formula,data=selected.data)
+   pred.glm        =  fitted(kapur.glm)
+   pred.glm        =  sortPredByLevels(pred.glm)
+   most.likely.glm =  apply(pred.glm,1,which.max)
+   tab.pred.facies =  table(most.likely.glm)
+   tot.obs         =   sum(tab.pred.facies)
+   all.fac.name    =  names(table(data$facies))
+   N.facies        =   length(all.fac.name)
+   idxToFac        =   all.fac.name
+   names(idxToFac) =   as.character(1:N.facies)
+   sel.fac.name    =   names(tab.pred.facies)
+   names(tab.pred.facies) = idxToFac[sel.fac.name]
+   no.occur.facies =   setdiff(all.fac.name,sel.fac.name)
+   prob            =   rep(NA,N.facies)
+   names(prob)     =  all.fac.name
+   if (any(as.integer(no.occur.facies))) {
+     prob[no.occur.facies] = 0
+   }
+   prob[sel.fac.name]  =  as.vector(tab.pred.facies)/tot.obs
+ return(prob)
+ }
\end{Sinput}
\end{Schunk}

Note that when we are using bootstraping for categorical data, we face two important challenges. First, some facies type may not be sampled and second some facies type may not be predicted. For both of these situations we should put the probability of such facies to zero. Here, facies 2 and 7 occur relatively few times and sometimes they may not be sampled or predictred. Below lines (which are in \texttt{prob.facies} function) are added for detecting unsampled facies and enforcing their probabilities as zero when bootstapping. Below chunck demonstrates the idea of naming  existing facies according to the indeices that come out of \texttt{which.max} function.

